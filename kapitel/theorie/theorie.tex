\section{Theorie}
\subsection{Stationärer Handel}
Der stationäre Einzelhandel bezeichnet den Verkauf von Waren an einem festen physischen Standort, wie zum Beispiel einer Filiale, einem Ladenlokal oder einem Handelsbetrieb. An diesem Ort bietet der Händler seine Produkte direkt zum Verkauf an und richtet sich vor allem an Kund:innen, die den Standort regelmäßig oder zufällig besuchen (BWL-Lexikon, o. J.). Ein wesentliches Merkmal des stationären Handels ist der direkte, persönliche Kontakt zwischen Kund:innen und den physischen Produkten. Diese können vor Ort begutachtet, ausprobiert und unmittelbar nach dem Kauf mitgenommen werden (Herting, 2024). 

\subsection{Online-Shopping}
Online-Shopping, auch als E-Commerce bezeichnet, beschreibt den digitalen Kauf und Verkauf von Waren und Dienstleistungen über das Internet (Statista, 2023). Diese Form des Handels ermöglicht es Konsument:innen, Produkte orts- und zeitunabhängig auszuwählen, miteinander zu vergleichen, zu bestellen und elektronisch zu bezahlen. 

Typische Vertriebskanäle sind Online-Shops sowie digitale Marktplätze wie Amazon oder Zalando. Diese Plattformen nutzen moderne technologische Systeme, welche wesentliche Prozesse wie Lagerverwaltung, Rechnungserstellung und Versandlogistik automatisieren und somit einen effizienten Ablauf sicherstellen (Marconomy, 2023). In Deutschland gehören Bekleidung sowie Elektronikartikel zu den beliebtesten Warengruppen im Onlinehandel (Statista, 2023).

\subsection{Konsumverhalten}
Wiswede (2000, S. 24) definiert Konsum wie folgt: 
„Der Begriff Konsum bezeichnet sämtliche Verhaltensweisen, die auf die Erlangung und private Nutzung wirtschaftlicher Güter und Dienstleistungen gerichtet sind. Diesen Prozess kann man unter verschiedenen Aspekten betrachten: als Vorgang der Einkommensverwendung, als Vorgang der Marktnahme oder als Vorgang der Nutzung dieser Güter durch den Konsumenten bzw. durch den Haushalt.“ 

Darauf aufbauend versteht die internationale Forschung unter dem Begriff „Consumer Behavior“ die Untersuchung von Menschen, Gruppen und Organisationen sowie aller Aktivitäten, die mit dem Kauf, der Nutzung und auch der Entsorgung von Waren und Dienstleistungen zu tun haben (Kale, Raj, Kumar \& Ranjan, 2023).  

Wie Gajjar (2013) beschreibt, ist das Konsumverhalten ein komplexes Phänomen, das von zahlreichen miteinander verknüpften Variablen beeinflusst wird. Diese lassen sich in drei Hauptkategorien einteilen: äußere Umweltfaktoren (wie Kultur, soziale Schicht, Religion oder wirtschaftliche Bedingungen), soziale Einflüsse (z. B. Familie oder Freundeskreis) sowie persönliche und psychologische Einflussgrößen. 

\subsection{Digitale Transformation}
Um das Konsumverhalten im digitalen Zeitalter besser zu verstehen, ist es hilfreich, theoretische Modelle heranzuziehen, die psychologische, soziale und ökonomische Einflussfaktoren erklären. Verschiedene Ansätze bieten strukturierte Erklärungsrahmen, um zu analysieren, wie Konsument:innen Entscheidungen treffen, welche Reize ihre Wahrnehmung beeinflussen und welche Rolle Routinen oder soziale Normen spielen. 

Im Folgenden werden drei etablierte Theorien vorgestellt, die besonders im Kontext des Online-Shoppings und der digitalen Transformation an Bedeutung gewonnen haben: 
das S-O-R-Modell, die Theory of Planned Behavior sowie verhaltensökonomische Ansätze zur Gewohnheitsbildung. 
\subsubsection{S-O-R Modell}
Das S-O-R-Modell ist ein theoretisches Rahmenkonzept aus der Psychologie und wird häufig zur Erklärung des Konsumentenverhaltens eingesetzt. Es basiert auf der Annahme, dass ein äußerer Reiz (Stimulus, S) auf einen inneren Verarbeitungsprozess im Menschen (Organismus, O) trifft, der schließlich zu einer beobachtbaren Reaktion (Response, R) führt (Mehrabian \& Russell, 1974). 

Während der Corona-Pandemie haben sich viele Menschen verstärkt dem Online-Shopping zugewandt. In dieser Umgebung wirken verschiedene Reize, wie beispielsweise Produktbewertungen, Werbebanner oder Empfehlungen, auf die Konsument:innen ein. Diese Reize beeinflussen vor allem das Vertrauen und die Zufriedenheit der Kund:innen. Forschungsergebnisse zeigen, dass die Qualität der Informationen und die soziale Präsenz von Online-Bewertungen das Vertrauen der Nutzer:innen deutlich stärken. Ein höheres Vertrauen führt dazu, dass die Kund:innen zufriedener sind und eine größere Bereitschaft zeigen, ein Produkt zu kaufen (Zhu et al., 2020). 

Im Gegensatz dazu spielt im stationären Einzelhandel die Atmosphäre des Ladens eine wichtige Rolle. Dort wirken verschiedene Reize wie Licht, Musik, Düfte und die Gestaltung der Verkaufsfläche auf die Gefühle der Kund:innen ein. Solche Reize können positive Emotionen wie Freude, Erregung oder Zufriedenheit hervorrufen. Diese emotionalen Reaktionen sind wichtig, da sie das Verhalten der Käufer:innen positiv beeinflussen und die Entscheidung zum Kauf unterstützen können (Jin et al., 2021; Loureiro et al., 2021, zitiert nach Erensoy et al., 2024).  

Insgesamt zeigt das S-O-R-Modell, dass unterschiedliche Reize in Online- und im stationären Handel verschiedene psychologische Prozesse und damit auch unterschiedliche Verhaltensweisen bei den Konsument:innen auslösen können. 