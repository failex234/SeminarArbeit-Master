\section{Vorwort}
Eventuelles Vorwort für Hinweis auf zusätzliche Textautoring


\section{Einleitung}
\subsection{Motivation und Relevanz der Thematik}
Die Digitalisierung hat in den letzten Jahren viele Bereiche unseres Lebens verändert. Menschen nutzen jeden Tag soziale Medien, Apps, Online-Plattformen und sogar Künstliche Intelligenz. Diese Technologien sind mittlerweile fester Bestandteil unseres Alltags. 

Ein Blick auf die Zahlen zeigt diese Entwicklung deutlich: Im Jahr 2004 lag der Umsatz im deutschen Onlinehandel noch bei etwa 4,4 Milliarden Euro. Im Jahr 2024 waren es bereits 88,8 Milliarden Euro (Handelsverband Deutschland, 2025). Das zeigt ein starkes Wachstum. Gründe dafür sind zum Beispiel Werbung, die auf persönliche Interessen abgestimmt ist, der schnelle und bequeme Einkauf von zu Hause aus, Vertrauen in Online-Anbieter und die vielen Bewertungen anderer Käufer:innen (Sahu, 2024). Diese Trends beeinflussen nicht nur die Kund:innen, sondern auch Unternehmen und deren Geschäftsmodelle und Marketingstrategien (Sahu, 2024). Durch das Zusammenspiel dieser Faktoren wird der Trend zum Online-Shopping immer weiter verbreitet. 

\subsection{Einfluss der Corona-Pandemie auf das Konsumverhalten}
Die Corona-Pandemie begann Ende 2019 in China und erreichte Anfang 2020 auch Europa (Taylor,2021). Durch Lockdowns, Kontaktbeschränkungen und geschlossene Geschäfte mussten viele Menschen anfangen, online einzukaufen (Gu, S.; Slusarczyk, B.; Hajizada, S.; Kovalyova, I.; Sakhbieva, A., 2021). Im Jahr 2020 stieg der Umsatz im deutschen Onlinehandel um 23 \% auf 73 Milliarden Euro. Das war das größte Wachstum innerhalb von zehn Jahren (Handelsverband Deutschland, 2021). 

Laut Schwab und Malleret (2020) brachte die Pandemie einen sogenannten „Reset“ des Alltags. Sie schreiben, dass das alte Geschäftsmodell in dieser Zeit funktional „gestorben“ sei. Viele neue Gewohnheiten aus der Pandemie seien heute selbstverständlich und könnten auch langfristig bleiben. Das betrifft auch das Einkaufsverhalten. 

Laut Handelsverband Deutschland (2021) haben 20 \% der Internetnutzer:innen während der Pandemie zum ersten Mal online eingekauft. Über die Hälfte dieser Menschen möchte auch in Zukunft lieber online einkaufen. Auch in ländlichen Gegenden wurde Online-Shopping immer beliebter (Sahu, 2024). 

Eine weitere Studie von Diaz-Gutierrez, Mohammadi-Mavi und Ranjbari (2023) zeigt, dass sich das Einkaufsverhalten während der Pandemie stark verändert hat. Dabei spielten das persönliche Gesundheitsrisiko, die eigene Einstellung zum Einkaufen und das frühere Einkaufsverhalten eine große Rolle. Interessant ist: Für Lebensmittel ersetzt Online-Shopping oft den Einkauf im Geschäft. Bei anderen Produkten ergänzt es eher den stationären Handel. Über 75 \% der Menschen, die während der Pandemie zum ersten Mal online eingekauft haben, möchten das auch nach der Pandemie weiterhin tun. Gleichzeitig sagen 63 \% bis 85 \% derjenigen, die seltener in Läden eingekauft haben, dass sie wieder genauso oft wie früher in Geschäfte gehen wollen (ebd.). 

Im Jahr 2022, also nach der Hochphase der Pandemie, ging der Umsatz im Onlinehandel erstmals wieder leicht zurück, nämlich um 2,5 \%, wobei der stationäre Einzelhandel weiterhin eine wichtige Rolle spielt (Handelsverband Deutschland, 2023). Außerdem zeigt sich, dass viele Menschen zunehmend dazu neigen, digitale Informationen zu nutzen, bevor sie eine Kaufentscheidung treffen (Sayyida, Sri, Sri \& Syarief, 2021). Dies verdeutlicht, dass während der Corona-Pandemie neue und unterschiedliche Einkaufsgewohnheiten entstanden sind. 

\subsection{Zielsetzung und Forschungsfrage}
Es lässt sich beobachten, dass sich das Online-Einkaufsverhalten deutlich in den Phasen vor, während und nach der Pandemie verändert hat. Diese Entwicklungen werfen die Frage auf, inwiefern sich das Konsumverhalten langfristig gewandelt hat. Bleiben Online-Käufe die bevorzugte Einkaufsform oder kehren Konsument:innen nach der Pandemie in den stationären Handel zurück? Welche subjektiven Motive, Einstellungen und Bewertungen stehen hinter diesen Veränderungen? 

Ziel dieser Arbeit ist es, das individuelle Konsumverhalten im Spannungsfeld zwischen Online-Shopping und stationärem Einzelhandel vor, während und nach der Corona-Pandemie qualitativ zu untersuchen. Dabei soll analysiert werden, wie sich Einkaufsroutinen, Präferenzen und Einstellungen im Verlauf der Pandemie gewandelt haben und welche Faktoren diesen Wandel beeinflusst haben. 

\subsection{Forschungsansatz und Relevanz}
Im Fokus steht eine qualitative Perspektive, um subjektive Erfahrungen und Deutungsmuster der Konsument:innen besser zu verstehen. Durch leitfadengestützte Interviews soll ein tiefer Einblick in individuelle Wahrnehmungen, Entscheidungsprozesse und Einstellungen gewonnen werden, die in quantitativen Studien oft nicht ausreichend erfasst werden. 

Die zentrale Forschungsfrage lautet: 

„Wie hat die Corona-Pandemie das individuelle Kaufverhalten zwischen Online-Shopping und stationärem Einzelhandel verändert?“ 

Diese qualitative Untersuchung soll dazu beitragen, das Verständnis für Konsumverhalten in Krisenzeiten zu vertiefen und Handlungsempfehlungen für Handel, Stadtentwicklung und Politik abzuleiten. 